\RequirePackage{xcolor}
\documentclass[a2]{sciposter} % hoja mayor
\usepackage{graphicx}
\usepackage{url,doi} % para manejar hipervinculos
\usepackage{hyperref} % hipervinculos activables en el PDF
\hypersetup{hidelinks} % que no tengan cajas de color las ligas
\usepackage[spanish]{babel}   
\usepackage[utf8]{inputenc}
\usepackage[sort&compress,numbers]{natbib} % para las citas

\leftlogo[1]{img/UANL.png}
\rightlogo[1]{img/FIME.png} 

\title{Mejora de algoritmo de\\reconocimiento de emociones}
\author{Alexander Espronceda Gómez,\\Cecilia Jael Aguilar Aranda$^\dagger$,\\Satu Elisa Schaeffer}
\institute {Posgrado en Ingeniería de Sistemas}
\email{$^\dagger$cjaelaguilar@gmail.com}



\begin{document}

\conference{Verano Científico FIME UANL 2021}

\maketitle

\section{Introducción}

De qué se trata el proyecto. Hipótesis y objetivos. Motivación,
justificación.

\section{Antecedentes}

Conceptos y notación indispensables para que tus lectores puedan
entender el resto del trabajo.

\section{Estado de arte}

Qué han hecho los demás sobre este tema (citar a publicaciones
científicas, de preferencia publicadas en revistas que tengan un DOI y
que por lo menos algunos sean de los últimos cinco años). Si son
libros, que tengan un ISBN. Evitar citar puros sitios web.

Área de oportunidad: qué exactamente este trabajo contribuirá encima
de lo que ya existe. {\textquestiondown}Qué tiene de
diferente/original/impacto?

\section{Solución propuesta}

Metodología, herramientas (qué en sí haces, cómo lo haces, con qué lo
haces).

\section{Experimentos}

Diseño, reportaje y análisis de los resultados de los experimentos.

\section{Conclusiones}

Qué se logró hacer; qué posibilidad de trabajo a futuro se tiene para
este trabajo.

\subsection*{Agradecimientos}

Organismos que otorgaron beca. Las demás personas que no son autores
que ayudaron en algo. El póster se preparó con
\url{https://www.overleaf.com/}.

\bibliography{poster} % de donde viene los datos de las refs
\bibliographystyle{plainnat} % en cual estilo se acomodan

\end{document}
